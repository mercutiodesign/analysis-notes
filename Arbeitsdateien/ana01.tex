\documentclass[12pt,a4paper,leqno]{article}
\usepackage[utf8x]{inputenc}
\usepackage{ucs}
\usepackage{amsmath}
\usepackage{amsfonts}
\usepackage{amssymb}
\usepackage{fancyhdr}


\pagestyle{fancy}
\fancyhead{}
\fancyfoot{}
\fancyhead[L]{Analysis I}
\fancyhead[R]{19. September 2011}
\fancyfoot[C] {\thepage}

\author{Martin Dreher}

\begin{document}
\part*{Vorlesung WS 2011 / 2012: Analysis I}

\section{Reelle und komplexe Zahlen}

\begin{description}
\item[Notation:]
\begin{align*}
\mathbb{N} &\textrm{ -- Menge der natürlichen Zahlen}\\
\mathbb{Z} &\textrm{ -- Menge der ganzen Zahlen}\\
\mathbb{Q} &\textrm{ -- Menge der rationalen Zahlen}\\
\mathbb{R} &\textrm{ -- Menge der reellen Zahlen}
\end{align*}

\item[Frage:] Welche Eigenschaften charakterisieren die reellen Zahlen?
\subsection*{Grundrechenarten:} 
\begin{description}
\item[Addition:] aus $x, y \in \mathbb{R}$ bilden wir $x+y\in \mathbb{R}$
\begin{itemize}
\item[(A1)] Kommutativgesetz: $x + y = y + x$
\item[(A2)] Assoziativgesetz: $x + (y + z) = (x + y) + z$
\item[(A3)] neutrales Element: $\exists 0 \in \mathbb{R} : \forall x \in \mathbb{R} : x + 0 = x = 0 + x$
\item[(A4)] inverse Elemente: $\forall x, y \in \mathbb{R} : \exists z \in \mathbb{R} : x + z = y = z + x$ \begin{flushright}
($z = y - x$)
\end{flushright}
\end{itemize}

\item[Multiplikation:] aus $x, y \in \mathbb{R}$ bilden wir $x\cdot y\in \mathbb{R}$
\begin{itemize}
\item[(M1)] Kommutativgesetz: $x \cdot y = y \cdot x$
\item[(M2)] Assoziativgesetz: $x \cdot (y \cdot z) = (x \cdot y) \cdot z$
\item[(M3)] neutrales Element: $\exists 1 \in \mathbb{R} : \forall x \in \mathbb{R} : x \cdot 1 = x = 1 \cdot x$
\item[(M4)] inverse Elemente: $\forall x \neq 0 : \forall y \in \mathbb{R} : \exists z \in \mathbb{R} : x \cdot z = y = z \cdot x$ \begin{flushright}
($z = \frac{y}{x}$)
\end{flushright}
\end{itemize}
\end{description}

\item Es gilt außerdem das \underline{Distributivgesetz}:
\begin{itemize}
\item[(D)] $\forall x, y, z \in \mathbb{R} : x \cdot (y + z) = x \cdot y + x \cdot z$
\end{itemize}

\item Weiterhin ist in der Menge $\mathbb{R}$ eine Teilmenge $\mathbb{R}_+$ ausgezeichnet, mit folgenden Eigenschaften:
\begin{itemize}
\item[(O1)] Für alle $x \in \mathbb{R}$ gilt genau eine der Aussagen: \\ 
$x \in \mathbb{R}_+ \textrm{ oder } x=0  \textrm{ oder } -x \in  \mathbb{R}_+$
\item[(O2)] Aus  $x \in \mathbb{R}_+ \textrm{ und } y \in \mathbb{R}_+ \textrm{ folgt } x + y \in \mathbb{R}_+$
\item[(O3)] Aus $x \in \mathbb{R}_+ \textrm{ und } y \in \mathbb{R}_+ \textrm{ folgt } x \cdot y \in \mathbb{R}_+$.
\end{itemize}

\item[Bem:] Eine Menge $M$ mit Operationen $+$ und $ \cdot $ sowie eine ausgez. Teilmenge $M_+$ mit den Eigenschaften (A1) -- (A4), (M1) -- (M4), D und (O1) -- (O3) heißt \underline{angeordneter Körperangeordneter Körper}. \\

\subsection*{Die Relation ,,$\mathbf{<}$"}
\item[Def:] Wir sagen	
$\begin{array}{l l}
	x < y,		& \textrm{falls } y - x \in \mathbb{R}_+ \\
	x \leq y, 	& \textrm{falls } y - x \in \mathbb{R}_+ \cup \lbrace 0 \rbrace
\end{array}$.

Die Relation $<$ hat folgende Eigenschaften:

\renewcommand{\theenumi}{\roman{enumi}}
\begin{enumerate}
\item Aus $x < y$ und $y < z$ folgt $x < z$.
\item Für alle $a \in \mathbb{R}$ gilt: aus $x < y$ folgt $x + a < y + a$.
\item Aus $x < y$ folgt $-x > -y$.
\item Aus $a > 0$ und $x < y$ folgt $ax < ay$.
\item Aus $a < 0$ und $x < y$ folgt $ax > ay$.
\item Für alle $x \neq 0$ gilt $x^2 = x  \cdot  x > 0$
\item Aus $x > 0$ folgt $x^{-1} > 0$
\item Aus $0 < x < y$ folgt $x^{-1} > y^{-1}$
\end{enumerate}
Beweis von vii. $(x^{-1})^{2}=x^{-2} > 0$ (nach vi.) $ x(x^{*2})>0$
\subsection*{Betrag einer reellen Zahl}

Sei $x \in \mathbb{R}$.

Wir definieren 
\begin{equation*}
|x| = 
	\begin{cases}
		x & \textrm{falls } x \geq 0\\
		-x & \textrm{falls } x < 0\\
	\end{cases}
\end{equation*}


Somit gilt für alle $x \in \mathbb{R}: |x| \geq 0$.

Es gelten:
\renewcommand{\theequation}{B\arabic{equation}}
\setcounter{equation}{0}
\begin{align}
	\label{eq:b1} |x \cdot y| &= |x| \cdot |y| \\
	\label{eq:b2} |x+y| &\leq |x|+|y| \\	
	\label{eq:b3} \big||x|-|y|\big| &\leq |x+y|
\end{align}

\item[Beweis:] \hfill

\begin{itemize}
\item[] Zu \eqref{eq:b1}: \\
	 Für $x, y \geq 0$ folgt die Aussage direkt aus der Def. i.a. gilt $x = \pm x_0$ und $y = \pm y_0$ mit $x_0, y_0 \geq 0$.
	 $$|x\cdot y| =  |\pm x_0\cdot y_0 | = |x_0 \cdot y_0| = |x_0|\cdot|y_0|=|x|\cdot|y|$$
	 \begin{flushright}$\square$\end{flushright}

\item[] Zu \eqref{eq:b2}: \\
	Wegen $ x \leq |x|$ und $y \leq |y|$ (folgt aus der Definition)
	$$x + y \leq |x| + |y|$$
	$$\textrm{wegen } \quad -x \leq |x| \\ \textrm{ und } -y \leq |y| \quad \textrm{ gilt}$$
	$$\begin{array}{r l}
		   -x - y = -(x+y)  &\leq |x| + |y| \\
		\Rightarrow |x + y| &\leq |x| + |y|	
	\end{array}$$
	 \begin{flushright}$\square$\end{flushright}


\item[] Zu \eqref{eq:b3}: dies folgt direkt aus \eqref{eq:b2}, denn  
\begin{align*}
	|x| = |x+y-y| &\leq |x+y|+|-y| = |x + y| + |y| \\
\textrm{d.h. } \qquad \qquad |x|-|y| &\leq |x+y|

!!!!!!!!!!!!HIER!!!!!!!!!!

zusammen folgt die Aussage von \eqref{eq:b3}.

\end{align*}
	 \begin{flushright}$\square$\end{flushright}


\end{itemize}

\subsection*{Ganzzahlige Potenzen reeller Zahlen:}
$$x \in \mathbb{R} \textrm{, } k \in \mathbb{N} \quad x^k = \underbrace{x  \cdot  x  \cdots  x}_{k mal} \quad x^0 := 1$$
$$\textrm{für } x \neq 0 : x^-1 = 1 / x \textrm{ und wir definieren für } k \in \mathbb{Z}, k < 0. x ^{k} := \frac{1}{x} ^{-k}$$

Dann gilt offenbar $$x^k  \cdot  x ^l = x ^{k+l}$$.

\subsection*{Wurzelziehen}

$$x \in \mathbb{R}, x > 0, k \in N$$

\item[Beh:] $∃y \in \mathbb{R}, y > 0$ mit $y ^ k = x$. Wir schreiben dann $$y = \sqrt[k]{x}$$
(Beweis am Freitag)

\newpage

Sei $A=(a_1,\dotsc, a_n)$ ein n-Tupel positiver reeller Zahlen $a_i \in \mathbb{R}_+$:

\begin{align}
	H_n(A) &:= \frac{n}{\frac{1}{a_1} + ... + \frac{1}{a_n}} & \textrm{harmonisches Mittel} \\
	G_n(A) &:= \sqrt[n]{a_1,\dotsc, a_n} & \textrm{geometrisches Mittel}\\
	A_n(A) &:= \frac{a_1 + ... + a_n}{n}  & \textrm{arithmetisches Mittel}\\
	Q_n(A) &:= \sqrt{\frac{a_1^2 + ... + a_n^2}{n}} & \textrm{quadratisches  Mittel}
\end{align}
\end{description}

\section*{Satz 1}

	Für jedes n-Tupel $A=(a_1,\dotsc, a_n)$ positiver reeller Zahlen gilt:
	$$\textrm{min}(a_1,\dotsc, a_n) \underset{\textrm{(1)}}{\leq} 
	 H_n(A) \underset{\textrm{(2)}}{\leq} 
	 G_n(A) \underset{\textrm{(3)}}{\leq} 
	 A_n(A) \underset{\textrm{(4)}}{\leq} 
	 Q_n(A) \underset{\textrm{(5)}}{\leq} 
	 \textrm{max}(a_1,\dotsc, a_n)$$
	
\begin{description}
\item[Bew:] \hfill

\begin{itemize}

	\item[] Zu (1): O.B.d.A $a_1 \leq a_2 \leq .. \leq a_n$, d.h. $a_1 =$ min$(a_1,\dotsc, a_n)$
	Dann gilt $\forall i = 1,\dotsc, n$:
	
	\begin{align*}
	& &a_1 &\leq a_i \\
	&\Rightarrow &\frac{1}{a_1} &\geq \frac{1}{a_i} \\
	&\Rightarrow &\frac{1}{a_1} + \ldots + \frac{1}{a_n} &\leq n  \cdot  \frac{1}{a_n} \\
	&\Rightarrow &H_n(A) = \frac{n}{\frac{1}{a_1} + \ldots + \frac{1}{a_n}} &\geq \frac{n}{\left(\frac{n}{a_1}\right)} = a_1 = \textrm{min}(a_1,\dotsc, a_n)
	\end{align*}
	 \begin{flushright}$\square$\end{flushright}

	\item[] Zu (3): Induktion nach $n$:
	$$n = 1: A_1(a_1) = a_1 = G_1(a_1)$$
	
	Ind-vor: Die Aussage gilt für alle $(n-1)$ - Tupel positiver reeller Zahlen.

	Ind-beh: Die Aussage gilt für alle $n$  - Tupel positiver reeller Zahlen.

	Sei $A=(a_1,\dotsc, a_n)$ ein $n$-Tupel positiver reeller Zahlen und\\
	$A' := (a_1,\dotsc, a_{n-1})$ ein $(n-1)$-Tupel positiver reeller Zahlen.


	Wir zeigen:
	\renewcommand{\theequation}{*} 
	\begin{equation}	
	A_n(A) - G_n(A) \geq \frac{n-1}{n}    \cdot  \bigg( A_{n-1}(A') - G_{n-1}(A')\bigg)
	\end{equation}
	
	Um (*) zu beweisen, zeigen wir zunächst:
	\renewcommand{\theequation}{**} 
	\begin{equation}	
		\textrm{Für jedes } x \geq 0 : x ^ n + n - 1 \geq n  \cdot  x
	\end{equation}

		\textbf{Bew.} von (**): $$x ^ n - nx + n -1 = (x-1)(x ^{n-1} + x^{n-2}+...+x-n-1)$$
		
		
			Für $x \geq 1$ sind beide Faktoren $\geq 0, \Rightarrow x ^ n - nx + n - 1 \geq 0$.
			
			Für $x \leq 1$ sind beide Faktoren $< 0, \Rightarrow x ^ n - nx + n - 1 < 0$.
			\begin{flushright}$ \square $ \end{flushright}\\

	Nun gilt aber	
		$$\frac{G_{n-1}(A')}{n} \left( (n-1)\cdot  \frac{A_{n-1}(A')}{G_{n-1}(A')} + \left( \frac{G_n(A)}{G_{n-1}(A')} \right) ^{\displaystyle n} \right)$$
$$= \frac{n - 1}{n} \cdot A_{n-1}(A') + \frac{1}{n} \cdot \frac{G_n(A)^n}{G_{n-1}(A')^{n-1}}$$
$$ = \frac{a_1 + ... + a_{n-1}}{n} + \frac{a_n}{n} = A_n(A)$$
Daraus folgt aber
$$A_n(A) = \frac{G_{n-1}(A')}{n} \cdot \left( (n-1) \cdot \frac{A_{n-1}(A')}{G_{n-1}(A')} + \left( \frac{G_n(A)}{G_{n-1}(A')} \right) ^{\displaystyle n} \right) $$
\reversemarginpar \marginpar{\tiny \begin{center}Benutzung von (**)\end{center} $$x = \frac{G_n(A)}{G_{n-1}(A')}$$}
$$\geq 	\frac{G_{n-1}(A')}{n}\cdot \left( (n-1) \cdot \frac{A_{n-1}(A')}{G_{n-1}(A')} + n \cdot \frac{G_n(A)}{G_{n-1}(A')} - n + 1 \right)$$
$$= \frac{n-1}{n} \cdot A_{n-1}(A') + G_n(A) - G_{n-1}(A')  \cdot   \frac{n-1}{n}$$
$$ \Leftrightarrow (*)$$
\begin{flushright}$\square$\end{flushright}


\begin{description}
\item[Bew (2):] z.z. $H_n(A) \leq G_n(A)$

Sei $A=(a_1,\dotsc, a_n)$ ein $n$-Tupel pos. reeller Zahlen.

Wir betrachten $\overline{A} = \frac{1}{a_1} + ... + \frac{1}{a_n}$. 


\begin{align*}
&\textrm{Wir wissen } & G_n(\overline{A}) &\leq A_n(\overline{A}) \\
&\Leftrightarrow & \sqrt[n]{\scriptstyle \frac{1}{a_1 + \ldots + a_n}} &\leq \frac{\scriptstyle \frac{1}{a_1} + \ldots + \frac{1}{a_n}}{n} \\
&\Leftrightarrow & \frac{n}{\scriptstyle \frac{1}{a_1} + \ldots + \frac{1}{a_n}} &\leq \sqrt[n]{a_1+ \ldots + a_n}
\end{align*}
\begin{flushright}$\square$\end{flushright}
\end{description}

\item[] Zu (4) z.z. $A_n(A) \leq Q_n (A)$

\begin{align*}
&\textrm{Wir wissen } &\forall x,y \in \mathbb{R}: && (x-y)^2 & \geq 0 \\
&& \Leftrightarrow &&x^2 - 2xy + y^2 &\geq 0 \\
&& \Leftrightarrow &&x^2 + y^2 &\geq 2xy
\end{align*}

Zu $A=(a_1,\dotsc, a_n)$ betrachten wir

\begin{align*}
&&(a_1 +\ldots + a_n)^2 &= a_1^2 + ... + a_n^2 + \underbrace{2a_1 a_2}_{\leq a_1^2+a_2^2} + \underbrace{2a_1 a_3}_{\leq a_1^2+a_3^2} + ... + \underbrace{2a_{n-1} a_n}_{\leq a_{n-1}^2+a_n^2} \\
&&& \leq n ( a_1^2 + ... + a_n^2 ) \\
&\Leftrightarrow &\left( \frac{a_1 + \ldots + a_n}{n} \right) ^ 2 &\leq \frac{a_1^2 + ... + a_n^2}{n} \\
&\Leftrightarrow &A_n(A) &\leq Q_n(A)
\end{align*}
\begin{flushright}$\square$\end{flushright}

Zu (5) z.z. $Q_n(A) \leq \textrm{max}(a_1,\dotsc, a_n)$

Sei wieder $A = (a_1,\dotsc, a_n) , \qquad a_1 \leq a_2 \leq \ldots \leq a_n$

$$Q_n(A) = \sqrt{\frac{a_1^2 + ... + a_n^2}{n}} \leq \sqrt{\frac{n \cdot a_n^2}{n}} = a_n = \textrm{max}(a_1,\dotsc, a_n)$$

\begin{flushright}$\square$\end{flushright}
\end{itemize}
\end{description}
\end{document}