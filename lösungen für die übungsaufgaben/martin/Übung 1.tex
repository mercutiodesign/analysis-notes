\documentclass[12pt,a4paper,leqno]{article}
\usepackage[utf8x]{inputenc}
\usepackage{ucs}
\usepackage{amsmath}
\usepackage{amsfonts}
\usepackage{amssymb}
\usepackage{ulsy}
\usepackage[german]{babel}
\usepackage{fancyhdr}

\pagestyle{fancy}
\fancyhead{}
\fancyfoot{}
\fancyhead[L]{Übungsaufgaben Analysis}
\fancyhead[C]{Martin Dreher}
\fancyhead[R]{26. September 2011}
\fancyfoot[C] {\thepage}

\author{Martin Dreher}

\begin{document}
\section{Übungsaufgaben}


\begin{enumerate}
\setcounter{enumi}{28}
\item	Beweisen Sie: Für $x,y\in \mathbb{R}$ gilt $|x+y|= |x|+|y|$ genau dann, wenn $x\cdot y \ge 0$.
\item	Lösen Sie folgende Ungleichungen für $x\in \mathbb{R}$:
		\begin{enumerate}
			\item[a)] $|x-2| \ge 10$
			\item[b)] $|x| > |x+1|$
			\item[c)] $|2x-1|<|x-1|$
		\end{enumerate}
\item	Beweisen Sie für alle $n \in \mathbb{N}$ die folgenden Aussagen:
		\begin{enumerate}
			\item[a)] $1^2 + 2^2 + 3^2 + \dots + n^2 = \frac {n (n+1)(2n+1)} 6$.
			\item[b)] ${n \choose 0} + {n \choose 1} + {n \choose 2} + \dots + {n \choose n} = 2^n$
		\end{enumerate}
	Im Teil (b) sind die {\em Binomialkoeffizienten} definiert als ${n \choose k} = \frac {n! }{k! (n-k)! }$, wobei man $0! :=1$ und für $n \ge 1$ rekursiv $n! :=(n-1)! \cdot n$ setzt, d.h. $n! $ ist das Produkt aller natürlichen Zahlen $\le n$.
\item	Beweisen Sie für alle $n \in \mathbb{N}$ die folgende Ungleichung:
	$$
	1 + \frac 1 {\sqrt{2}} + \frac 1 {\sqrt{3}} + \dots + \frac 1 {\sqrt{n}} > 2\left(\sqrt{n+1}-1\right).
	$$
\item	Besitzt die Menge $M= \{ \frac m n \,: \, m,n \in \mathbb{N}, 0<m<n \}$ ein größtes oder ein kleinstes Element? Beweisen Sie Ihre Behauptung!
\end{enumerate}

\section{Lösungen}


\begin{enumerate}
\setcounter{enumi}{28}

\item %Frage 29%
\begin{description}
\item[Behauptung:] Für $x,y\in \mathbb{R}$ gilt 
	\begin{equation*}
	|x+y|= |x|+|y| \quad \Leftrightarrow \quad x\cdot y \ge 0
	\end{equation*}		
\item[Beweis:] \hfill \\
	Es wird eine Fallunterscheidung getroffen.
	
	Fall 1: Seien $x, y \geq 0$, so kann man schreiben
	\begin{displaymath}
		|x + y| = x + y = |x| + |y|
	\end{displaymath}
	
	
	Fall 2: Seien $x, y < 0$, so kann man schreiben
	\begin{displaymath}
		|x + y| = -x + -y = |x| + |y|
	\end{displaymath}
	
	Fall 3: Wenn $x\cdot y<0$ so gilt $|x+y| = |x|+|y|$ \textbf{nicht}. \\
	\textbf{Beweis}: O.B.d.A. sei $x < 0$ und $y > 0$, so gilt
	\begin{displaymath}
		|x+y| = - (x+y) \neq -x + y = |x| + |y| 
	\end{displaymath}
	
	\begin{flushright}$\square$\end{flushright}
	
	

\end{description}
\item %Frage 30% 



\begin{enumerate}
\item[a)] 
	\begin{displaymath}
	\begin{array}{l l}
	|x-2| \geq 10 &\Leftrightarrow \left\{
	\begin{array}{l l}
		x \geq 10 & \textrm{falls } x \geq 2 \\[6pt]
		x \leq -12 & \textrm{falls } x < 2 \\
	\end{array}\right.  \\ 
	\hfill \\
	& \Leftrightarrow x \leq -12 \vee x \geq 10 
	\end{array}
	\end{displaymath}
\item[b)] 

	\begin{displaymath}
	\begin{array}{l l}
		|x| > |x+1| &\Leftrightarrow  \left\{
	\begin{array}{l l}
		  0 > 1 \textrm{\blitza} & \textrm{für } x \geq 0 \\[6pt]
		  x < -\frac{1}{2} & \textrm{für } -1 \leq x < 0 \\[6pt]
		  0 < 1 & \textrm{für } x < -1
	\end{array}\right.  \\ 
	\hfill \\
	& \Leftrightarrow x <  -\frac{1}{2} 
	\end{array}
	\end{displaymath}
	
\item[c)] 
	\begin{displaymath}
	\begin{array}{l l}
		|2x-1|<|x-1| &\Leftrightarrow  \left\{
	\begin{array}{l l}
		x < 0 & \textrm{für } x \geq 1 \\[6pt]
		x < \frac{2}{3}  & \textrm{für } \frac{1}{2} \leq x < 1 \\[6pt]
		x > 0  & \textrm{für } x < \frac{1}{2}
	\end{array}\right.  \\ 
	\hfill \\
	& \Leftrightarrow \frac{1}{2} \leq x < \frac{2}{3}
	\end{array}
	\end{displaymath}
\end{enumerate}

\item %Frage 31% 


\begin{enumerate}
\item[a)] \hfill


\begin{description}
\item[Beh:] $$1^2 + 2^2 + 3^2 + \ldots + n^2 = \frac {n (n+1)(2n+1)} 6$$
\item[Bew:] Die Behauptung wird durch vollständige Induktion über $n$ gezeigt.

\textbf{Ind.-Ansatz: }
$$n = 1 : \quad 1 ^2 = 1 = \frac{1\cdot 2\cdot 3}{6}$$

\textbf{Ind.-Beh:}
Für $n>1$ gelte:
$$1^2+2^2+\ldots+ (n-1)^2 = \frac{(n-1)(n)(2n -1)}{6}$$


\textbf{Ind.-Schritt: }
Aus der Gültigket für $n-1$ wird auf die Gültigkeit für $n$ geschlossen, für alle $n > 1$:
\begin{align*}
1^2+2^2+\ldots+ (n-1)^2 + n^2 &=  \frac{(n-1)(n)(2n -1)}{6} + n^2\\[12pt]
\frac{n(n-1)(2n-1)+6n^2}{6} &= \frac{(2n^3-n^2-2n^2+n)+6n^2}{6} \\[12pt]
\frac{2n^3+3n^2+n}{6} &=\frac{n(n+1)(2n+1)}{6}
\end{align*}
\begin{flushright}$\square$\\[12pt]\end{flushright}
	
\end{description}

			
\item[b)] 

\begin{description}
\item[Beh:] $${n \choose 0} + {n \choose 1} + {n \choose 2} + \dots + {n \choose n} = 2^n$$
\item[Bew:] Die Behauptung wird durch vollständige Induktion über $n$ gezeigt.

\textbf{Ind.-Ansatz: }
$$n = 0 : \quad {0 \choose 0} = \frac{0!}{0! \cdot 0!} = 1 = 2^0$$

\textbf{Ind.-Beh:}
Es gelte:
$${n \choose 0} + {n \choose 1} + {n \choose 2} + \dots + {n \choose n} = 2^n $$

\textbf{Ind.-Schritt: }
Aus der Gültigket für $n$ wird auf die Gültigkeit für $n+1$ geschlossen:
\begin{align*}
&{n + 1 \choose 0} + {n + 1 \choose 1} + {n + 1 \choose 2} + \dots + {n+1 \choose n-1} + {n + 1 \choose n + 1} = 2^n + {n + 1 \choose n + 1} \\[12pt]
 & 2^{n-1} + \frac {n!}{n!} = 2^{n+1} 
\end{align*}

\end{description}

\end{enumerate}


\end{enumerate}

\end{document}
