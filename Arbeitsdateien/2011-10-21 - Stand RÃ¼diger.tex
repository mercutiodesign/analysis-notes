\documentclass[12pt,a4paper]{article}
\usepackage[utf8x]{inputenc}
\usepackage{ucs}
\usepackage[leqno]{amsmath}
\usepackage{amsfonts}
\usepackage{amssymb}
\usepackage{ulsy}
\usepackage{fancyhdr}
\pagestyle{fancy}
\fancyhead{}
\fancyfoot{}
\fancyhead[L]{Analysis I}
\fancyhead[R]{21. September  2011}
\fancyfoot[C]{\thepage}

\author{ Rüdiger Brecht }

\begin{document}
\part*{Analysis }


\subsection*{Satz 2:} Für l,k $ \in \mathbb{N} $ gilt $\sqrt[n]{k}\in \mathbb{Q} \Leftrightarrow \sqrt[l]{k}\in \mathbb{N}$


\begin{description}
\item[Bew:] $"\Leftarrow "$ offensichtlich

$" \Rightarrow "$ ist $\sqrt[l]{k} \in \mathbb{Q}$ so existiert m,n $\in \mathbb{N}$ mit 
$\sqrt[l]{k}=\frac{m}{n}$ 
\\
\\
o.B.d.A gilt $ggT(m,n)=1$ 

wir wollen zeigen n=1
\\
\\
indirekter Beweis: Wir nehmen an $n>1$
\\
\\
$\Rightarrow \quad \exists$Primzahl $q$ mit $q|n$

$\sqrt[l]{k}=\frac{m}{n}\Rightarrow n^{l}k=m^{l}
q|n$ $\Rightarrow$ $q|n^{l}k = m^{l}$

$\Rightarrow q|n$ und $q|m $ \blitze zu $ggT(m,n)=1$

$\Rightarrow n=1$

$\Rightarrow \sqrt[l]{K}=m \in \mathbb{N} $
\begin{flushright} $ \square $ \end{flushright}
\end{description}
---
\subsection*{Satz 3:} Für jede reelle Zahl $x>0$ und jedes $n\in \mathbb{N}$ existiert genau eine reelle Zahl $y>0$ mit $y^{n}=x$.
Diese bezeichnen wir mit $\sqrt[n]{x}=x^{\frac{1}{n}}$

\begin{description}
\item[Bew:] Zunächst bemerken wir:

aus $0<y_{1}<y_{2}$ folgt, dass $y_{1}^{n}<y_{2}^{n}$
$\Rightarrow$ es gibt höchstens eine Lösung der Gleichung $y^{n}=x$

Wir zeigen nur noch die Existenz
\subsection*{Schritt 1:} Für alle $0<a<b$ und alle $n \in \mathbb{N}$ gilt
\renewcommand{\theequation}{*}
\begin{equation}
b^{n}-a^{n}<n(b-a)b^{n-1}
\end{equation}
\end{description}
\begin{description}
\item[Bew:] $$b^{n}-a^{n}=(b-a)(b^{n-1}+\underbrace{b^{n-2}a}_{<b^{n-1}}+\cdots+\underbrace{ba^{n-2}}_{<b^{n-1}}+\underbrace{a^{n-1}}_{b^{n-1}})$$ 

$$<n(b-a)b^{n-1}$$
\begin{flushright} $ \square $ \end{flushright}
\end{description}
\subsection*{Schritt 2:} Die Menge
$A:=\lbrace t\in \mathbb{R}:t\leq0 $ oder $ t>0 $ und $t^{n}<\overbrace{x}^{Satz 3}\rbrace$
hat kein größtes Element
\begin{description}
\item[Bew:] (a) A enthält positive Zahlen, zum Beispiel $t=\frac{x}{x+1}$

denn $0<t<1$ 

Falls also A ein größtes Element hätte, so wäre dies positiv.

Sei nun $t_{0}>0$ ein Element von A 

$\Rightarrow x-t_{0}^{n}>0 \Rightarrow \frac{x-t_{0}^{n}}{n(t_{0}^{n}+1)^{n-1}}>0$  

Wir wählen nun $h \in\mathbb{R}$ mit $0<h<1$
$$ h<\frac{x-t_{0}^{n}}{n(t_0^n+1)^{n-1}}$$
z.B. $h=min(\frac{1}{2},\frac{1}{2} \frac{x-t_{0}^{n}}{n(t_{0}^{n}+1)^{n-1}} )$

Wir betrachten jetzt $t_{1}:=t_{0}+h$
$$ t_{1}^{n}-t_{0}^{n}\overbrace{<}^{(*)}n h (t_{0}+h)^{n-1}<n h (t_{0}+1)^{n-1}<x-t_{0^{n}}$$

$\Rightarrow t_{0}^{n}<x \Rightarrow t_{1} \in A$

$\Rightarrow t_{0}$ war nicht größtes Element von A

Weil dies für jedes positive Element von A gilt, hat A kein größtes Element. 
\end{description}
---
Nach Schritt 2 hat A kein größtes Element $\Rightarrow$ B hat  kein kleinstes Element. Wir nennen dies y.

\begin{description}
\item[Beh:] $y^{n}=x$. Wäre $y^{n}\neq x$ , so müsste $y^n>x$ gelten, und $y$ wäre kleinstes Element von C \blitze zu Schritt 3 $\Rightarrow y^n=x$
\begin{flushright}$ \square $ \end{flushright} 
\end{description}
\begin{description}
\item[Def:] Sei $ A\subseteq \mathbb{R}$

A heißt \underline{von oben beschränkt}, falls es eine Zahl $M$ gibt mit $\forall a \in A:a\leq M$ 

A heißt \underline{von unten beschränkt}, falls es eine Zahl $m$ gibt mit $\forall a \in A:m\leq a$
A heißt \underline{beschränkt}, falls A von oben und unten beschränkt ist.
\end{description}

\subsection*{Satz 4:} (1)
Sei $A\subseteq \mathbb{R}$ eine nicht leere von oben beschränkte Teilmenge.\\ Dann existiert genau eine Zahl $M_{0}$ mit folgenden Eigenschaften.\\
 
(i) $\forall a \ in A : a\leq M_{0}(M_{0}ist obere Schranke)$\\

(ii) $\forall \varepsilon>0\quad \exists a\in A: M_{0}-\varepsilon<a$\\

Diese Zahl $M_{0}$ nennen wir das Supremum von A, $M_{0}= Suo(A)$\\

\begin{flushleft}
(2) Sei A eine nicht leere von unten beschränkte Teilmenge.
\end{flushleft} 

 Dann existiert eine eindeutige Zahl $m_{0}\in \mathbb{R}$ mit\\

(i)$\forall a \ in A : m_{0}\leq(m_{0}ist obere Schranke)$ \\

(ii)$\forall \varepsilon>0\quad \exists a\in A: m_{0}+\varepsilon>a$\\

Diese Zahl nennt man das Infimum von A, $m_{0}= inf(a)$

---


Also ist $M_{0}$ das kleinste Element von X.\\

$\Rightarrow M_{0}$ ist obere Schranke von A, d.h. es gilt (i)\\

DA für jedes $\varepsilon > 0$  die Zahl $M_{0}-\varepsilon\notin X $ folgt:\\

$\forall \varepsilon >0 \quad \exists a \in A :M_{0}-\varepsilon <a$ d.h. Eigenschaft (ii)

$M_{0}$ ist eindeutig:\\

Wären $M_{0}$ und $M_{1}$ zwei Zahlen mit den Eigenschaften (i) und (ii)\\

$\Rightarrow M_{0} $und$ M_{1}$ wären kleinste Schranken von A\\

$\Rightarrow M_{0} \leq M_{1}$und$M_{1}\leq M_{0}\Rightarrow M_{0}=M_{1}$
\begin{flushright}$ \square $ \end{flushright}

Potenzen $a^{x}$: Sei $a0>$

Für $k\in \mathbb{Z}$ haben wir $a^{k}$ bereits definiert

ist $x=\frac{m}{n}>0$ , so definieren wir 
$$a^{x}:=\sqrt[n]{a^{m}}$$
Für $x\in \mathbb{Q},x<0$ definieren wir
$$a^{x}=\frac{1}{a^{-x}}$$
Dann gelten die üblichen Regeln 
$$a^{x_{1}+x_{2}}=a^{x_{1}}a^{x_{2}}$$
$$(a^{x_{1}})^{x2}=a^{x_{1}x_{2}}$$
für $x_{1},x_{2} \in \mathbb{Q}$   
 


\end{document}