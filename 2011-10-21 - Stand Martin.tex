\documentclass[12pt,a4paper,leqno]{article}
\usepackage[utf8x]{inputenc}
\usepackage{ucs}
\usepackage{amsmath}
\usepackage{amsfonts}
\usepackage{amssymb}
\usepackage{ulsy}
\usepackage{fancyhdr}

\pagestyle{fancy}
\fancyhead{}
\fancyfoot{}
\fancyhead[L]{Analysis I}
\fancyhead[R]{21. September  2011}
\fancyfoot[C]{\thepage}

\author{Martin Dreher}

\begin{document}
\part*{Analysis 1}
\ldots

nach Satz 2:

\section*{Reelle Zahlen}

(Intuitiv: Jeder Punkt liegt auf der Geraden)

\begin{description}
\item[Def:] Ein \underline{dedekindscher Schnitt} in den reellen Zahlen ist eine Zerlegung $\mathbb{R} = A \cup B$ in zwei nicht leere Teilmengen, so dass jede Zahl $a \in A$ kleiner ist als jede Zahl $b \in B$.
\end{description}

\hfill \\

\underline{Die Stetigkeitseigenschaft der reellen Zahlen (Dedekindsche Schnittaxiom)}

\hfill

Ist $\mathbb{R} = A \cup B$ ein dedekindscher Schnitt, so besitzt entweder $A$  eine größte Zahl oder $B$ besitzt eine kleinste Zahl. 

Anm: A und B sind disjunkt, da 

$$\mathbb{R} = A \cup B \textrm{(ein D-Schnitt)} \Leftrightarrow \forall a \in A \quad \forall b \in B : a < b$$


\ldots

jetzt folgt Satz 3: (Notiz: Bew. Für Schritt 1 per vollst. Induktion wäre interessant)


nach Schritt 2 von Satz 3:

\ldots

\subsection*{Schritt 2}

Die Menge 
\begin{displaymath}
C := { t \in \mathbb{R}: t > 0, t^n>x}
\end{displaymath}
hat kein kleinstes Element.

\begin{description}
\item[Bew:] Sei $t_0 \in C$ beliebig. Wir definieren 

$$h := \frac{t_0^n-x}{n \cdot t_0^{n-1}} < \frac{t_0^n}{nt_0^{n-1}} = \frac{t_0}{n} \leq t_0$$


Sei $t_1 := t_0 - h > 0$

Außerdem gilt: 
$$t_0^n - t_1^n \underbrace{<}_{\textrm{siehe (*)}} n \cdot h \cdot t_0^{n-1} = t_0^n - x \Rightarrow t_1^n > x \Rightarrow t_1^n \in B$$

\end{description}

\begin{flushright} $\square$ \end{flushright}

\subsection*{Schritt 4:}

Wir beweisen die Existenz von $y > 0$ mit $y^n = x$.

\begin{description}

\item[Bew:] Sei $A$ wie im Schritt 2 und sei $B = \mathbb{R} \backslash A$.

Aus der Definition von A folgt das 
$$ B = { t \in \mathbb{R} : t > 0 \textrm{ und } t^n \geq x } $$

$A \cup B$ ist ein dedekindscher Schnitt: gäbe es ein $a \in A$ und $b \in B$ mit $a \geq b \Rightarrow a > 0$. Dann folgt aus $a \geq b$ aber $a^n \geq b^n$. \blitze{} zu $a^n < x$ und $b^n \geq x$ 
 
$ \Rightarrow A \cup B$  ist ein dedekindscher Schnitt.

Nach Schritt 2 hat $A$ kein größtes Element $\Rightarrow B$ hat ein kleinstes Element. Wir nennen dieses $y$.

\item[Beh:] $y^n = x$. Wäre $y^n \neq x$, so müsste $y^n > x$ gelten, und $y$ wäre kleinstes Element von $C$. Widerspruch zu Schritt 3 $\Rightarrow y^n = x$
\begin{flushright} $ \square $ \end{flushright}
\end{description}

---

\ldots

Nun folgt Definition von Schranken, Satz 4.

\ldots

\subsection*{Beweis von Satz 4}

\begin{description}

\item[Bew:] 

Wir beweisen nur (1), da (2) völlig analog bewiesen wird.

Sei also $A$ nicht leer und von oben beschränkt.

Wir betrachten $X:= { M \in \mathbb{R} : \forall a \in A : a \leq M }$

(Notiz: $X \neq \emptyset$ nach Voraussetzung.)

Sei $Y := \mathbb{R} \backslash X$. Dann gilt $Y \neq \emptyset$, da $a - 1 \in Y$ für jedes $a \in A$.
Außerdem gilt: 
\begin{itemize}
\item[(a)] $\mathbb{R} = Y \cup X$
\item[(b)] für $y \in Y$ und $x \in X$ gilt stets $y < x$
\end{itemize}

Ist nämlich $y \in Y$, so ist $y$ \underline{nicht} obere Schranke von $A$

$$
\left.
\begin{array}{l}
\Rightarrow \exists a \in A \quad y < a \\
\textrm{weil } x \in X \quad a \leq x
\end{array}
\right\} \Rightarrow y < x. $$

(a), (b) $\Rightarrow \mathbb{R} = Y \cup X$ ist ein dedekindscher Schnitt.

Sei $M_0$ das kleinste Element von $X$ oder das größte Element von $Y$.

\item[Beh:] $M_0$ ist nicht größtes Element von $Y$.

Andernfalls wäre $M_0 \in Y \Rightarrow \exists a \in A : M_0 < a$  und für $M_1 = \frac{M_0 + a}{2}$ gilt $M_0 < M_1 < a$ Widerspruch zu $M_0$ ist größtes Element von $Y$.
\end{description}



\ldots

Nun folgt der Rest des Beweises von Satz 4

\underline{Potenzen $a^x$}

\ldots


Für $x \in \mathbb{R} \backslash \mathbb{Q}$ betrachten wir zwei Fälle:

Fall 1: ($a \geq 1$)

Sei $x \in \mathbb{R}$ gegeben.


$M(a, x) = { a^q | q \in \mathbb{Q} : q < x }$

Diese Menge ist von oben beschränkt durch $a^{\lfloor x \rfloor + 1}$ % Ganzzahliger Anteil von x%

Nach Satz 4 besitzt $M(a, x)$ ein Supremum, und wir definieren 
$$a^x = \textrm{sup } M(a,x)$$

Für $0 < a < 1$:

$$a^x := \left( \frac{1}{a} \right) ^{-x}$$


\end{document}
