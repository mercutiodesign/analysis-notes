\documentclass[12pt,a4paper,leqno]{article}
\usepackage[utf8x]{inputenc}
\usepackage{ucs}
\usepackage{amsmath}
\usepackage{amsfonts}
\usepackage{amssymb}
\usepackage{ulsy}
\usepackage{fancyhdr}


\pagestyle{fancy}
\fancyhead{}
\fancyfoot{}
\fancyhead[L]{Übungen Analysis I}
\fancyhead[R]{24.10.2011}
\fancyfoot[C] {\thepage}

\author{Rüdiger Brecht}
\begin{document}
\part*{Übungsblatt 1}
\subsection*{Aufgabe 29}
Beweisen Sie: Für $x,y \in \mathbb{R}$ gilt $|x+y|=|x| + |y|$ genau dann, wenn $x \cdot y \geq 0$.
\begin{description}
\item[Beh:]\forall $x,y \in \mathbb{R}$ gilt $|x+y|=|x| + |y|$ \Leftrightarrow $x \cdot y \geq 0$.
\item[Bew:] Für den Beweis werden wir vier Fallunterscheidungen vornehmen:
	\begin{description}
		\item[Fall 1:] Sei $x,y \leq 0$
						$$x + y = |x + y| = |x| + |y|$$ 
						\begin{flushright}$ \square $ \end{flushright}
						
		\item[Fall 2:] Sei $x,y < 0$
						$$-x + (-y) = -(x+y) = |-(x+y)| = |x + y| = |x| + |y|$$		
						\begin{flushright}$ \square $ \end{flushright}
		\item[Fall 3:] Sei o.B.d.A $x<0 , y>$
		        $$-x + y = -(x-y) = |-(x-y)| = |x-y| \neq |x+y| $$
		        \begin{flushright}$ \square $ \end{flushright}
		 \item[Fall 4:] Sei $x$ oder $y$ gleich $0$. Wählen wir o.B.d.A $x = 0$ und $y \neq 0$
		 		 $$0 \pm y = |\pm y + 0|= |y| = |y| + |0|$$
		 		 \begin{flushright}$ \square $ \end{flushright}       					
	\end{description}
\end{description} 

\subsection*{Aufgabe 30}
Lösen Sie folgende Ungleichungen für $x \in \mathbb{R}$
\begin{description}
	\item[a)] $|x-2|\geq10 $ 
	
			Für $(x-2)\geq 0 \Leftrightarrow x \geq -2 \Rightarrow x  \geq 12$ \\
			Für $ (x-2) < 0 \Leftrightarrow x < -2 \Rightarrow  x \geq -12$ \\
			$\Rightarrow L=$
			
	\item[b)] $ |x| >|x+1|$
	
			Für $(x+1) \geq 0 \Leftrightarrow x \geq -1 \Rightarrow x > x+1 $ \blitze   wegen $ 0>1 $ \\    				Für $x<0 \Rightarrow -x > -x+1 $  \blitze   wegen $ 0>1 $ \\
						$ \Rightarrow L = \emptyset$
						
	\item[c)] $|2x -1|<|x-1| \Leftrightarrow |2x -1|-|x-1|<0 $ 
	
			Für $(2x-1)\geq 0 \Leftrightarrow x\geq \frac{1}{2} \Rightarrow 2x-1-(x-1)<0 \Leftrightarrow x<0$ \blitze \\ aufgrund der Annahme $x \geq \frac{1}{2}$\\			Für	$(x-1)<0 \Leftrightarrow x < 1 \Rightarrow -2x-1-(-x-1)<0 \Leftrightarrow x<0$	
			$\Rightarrow L =\lbrace x \in \mathbb{R}:x<0\rbrace $
\end{description} 

\subsection*{Aufgabe 31}
Beweisen Sie für alle $n \in \mathbb{N}$ die folgende Aussagen:
\begin{description}
\item[a)] $1^2+2^2+3^2+\cdots+n^2=\frac{n(n+1)(2n+1)}{6}$
\begin{description}
	\item[Induktionsanfang: ]Wir zeigen die Gültigkeit der Formel für $n = 1$:
	$$\sum \limits _{j =1}^1 1^2= \frac{1(1+1)(2+1)}{6}=1$$
							Damit ist der Induktionsanfang gemacht.
	\item[Induktionsannahme:] $\sum \limits _{j =1}^n j^2 = \frac{n(n+1)(2n+1)}{6}$ für $n \in \mathbb{N}$
	\item[Induktionsschritt: ]Unter der Voraussetzung das die Induktionsannahme gilt, zeigen wir nun, dass sie auch für $(n+1)$ gilt:
		\begin{align*} \sum \limits _{j =1}^n j^2 + (n+1)^2 &\overset {I.A.}{=} \frac{n(n+1)(2n+1)}{6} + (n+1)^2 \\&= \frac{n(n+1)(2n+1)+6(n+1)^2}{6} \\ &=\frac{(n+1)[n(2n+1)+6(n+1)}{6} \\ &\overset {wegen (*)}{=}\frac{(n+1)(n+2)(2n+3)}{6} \end{align*}
	\renewcommand{\theequation}{*}
	 \begin{equation}
	 (n+2)(2n+3)=2n^2+7n+6=
	 \end{equation}
	 $$2n^2+2+6n+6=n(2n+1)+6(n+1)$$
	\begin{flushright}$ \square $ \end{flushright} 
	\newpage
\end{description}
	\item[b)]$\binom {n}{0}+\binom {n}{1}+\binom {n}{2}+\cdots + \binom {n}{n}=2^n $
		\begin{description}
			\item[Induktionsanfang: ] Wir zeigen die Gültigkeit für $n=1$:
			$$\sum \limits _{j =0}^1 \binom {n}{j} = \binom {1}{0}+ \binom {1}{1}=2=2^1=2$$
			\item[Induktionsannahme: ]Die Aussage gilt für alle $(n-1) \in \mathbb{N}$
			\item[Induktionsschritt: ]Unter der Voraussetzung das die Induktionsannahme gilt, zeigen wir nun, dass sie auch für $n$ gilt:
			\begin{align*} \sum \limits _{j =0}^n \binom {n-1}{j} = 2^{n-1} &= \sum \limits _{j =0}^n\frac{(n-1)!}{j!(n-j-1)!} \\&=  \sum \limits _{j =0}^n\frac{(n-1)!(n-j)}{j!(n-j)!} \\&= \frac{1}{2}\sum \limits _{j =0}^n\frac{(n-1)!(n-j)+(n-1!)(n-(n-k))}{j!(n-j)!} \\&=  \frac{1}{2} \sum \limits _{j =0}^n\frac{(n-1)!(n-j+n-n+j)}{j!(n-j)!} \\&= \frac{1}{2} \sum \limits _{j =0}^n\frac{(n)!}{j!(n-j)!} = 2^{n-1}=2^n \cdot \frac{1}{2} \\&=  \sum \limits _{j =0}^n\frac{(n)!}{j!(n-j)!}=2^n\end{align*}
			\begin{flushright}$ \square $ \end{flushright}
		\end{description}
\end{description}

\subsection*{Aufgabe 32}
Beweisen sie für alle $n \in \mathbb{N}$ die folgende Ungleichung:
$$ 1+ \frac{1}{\sqrt{2}}+\frac{1}{\sqrt{2}}+\frac{1}{\sqrt{3}}+ \cdots + \frac{1}{\sqrt{n}}> 2(\sqrt{n+1}-1)$$
\begin{description}
	\item[Beh: ]$ \sum \limits _{j =1}^n \frac{1}{\sqrt{j}}>2(\sqrt{n+1} -1) $ gilt für alle $n \in \mathbb{N}$
	\item[Bew: ]
	$$ 1+ \frac{1}{\sqrt{2}}+\frac{1}{\sqrt{2}}+ \cdots + \frac{1}{\sqrt{n}}> 2(\sqrt{n+1}-1) $$
	$$ \frac{1+ \frac{1}{\sqrt{2}}+\frac{1}{\sqrt{2}}+ \cdots + \frac{1}{\sqrt{n}}}{n} > \frac{2(\sqrt{n+1} -1)}{n} $$
	$A:=(1,\frac{1}{\sqrt{2}} , \ldots , \frac{1}{\sqrt{n}})$
	$$A_{n}(N)> \frac{2(\sqrt{n+1} -1)}{n} $$
	$A_{n}(A)\geq min(A)$
	$$1 \geq \frac{2(\sqrt{n+1} -1)}{n}$$
	$$n+2 \geq 2(\sqrt{n+1}) \Leftrightarrow n^2 +4n+4 \geq 4n+ +4 $$
	$$n^2 \geq 0 $$ \begin{flushright}da $n \in \mathbb{N}$ \end{flushright}
\begin{flushright}$ \square $ \end{flushright}
  
\end{description}
\end{document}
